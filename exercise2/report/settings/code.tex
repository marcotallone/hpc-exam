% Listings package
\usepackage{listings}

% float package to enforce figure positioning
\usepackage{float}

% New colors for syntax highlighting
\usepackage{newtxtt}

% Custom highlighting for different languages
% You can ADD A NEW LANGUAGE by copying the adding
% a new \lstdefinestyle{<language>} block below

\lstdefinestyle{Pseudocode}{
    language=Python,
    basicstyle=\color{black}\ttfamily\small,
    numberstyle=\color{gray},
    xleftmargin=5mm,
    numbers=left,
    numbersep=5mm,
    breakatwhitespace=false,         
    breaklines=true,                 
    captionpos=b,                    
    keepspaces=true,                 
    showspaces=false,                
    showstringspaces=false,
    showtabs=false,                  
    tabsize=2,
    morekeywords={function, if, else, return},
    mathescape=true
    % escapeinside={(*}{*)} % everithung between (* and *) won't be considered as code
}

\lstdefinestyle{Python}{
    language=Python,
    basicstyle=\color{black}\ttfamily\small,
    commentstyle=\color{gray},
    keywordstyle=\color{gitred},
    stringstyle=\color{gitlightblue},
    numberstyle=\color{gray},
    xleftmargin=10mm,
    numbers=left,
    numbersep=10mm,
    breakatwhitespace=false,         
    breaklines=true,                 
    captionpos=b,                    
    keepspaces=true,                 
    showspaces=false,                
    showstringspaces=false,
    showtabs=false,                  
    tabsize=2
}
\lstdefinestyle{Python}{
    language=Python,
    basicstyle=\color{black}\ttfamily\small,
    commentstyle=\color{gray},
    keywordstyle=\color{gitred},
    stringstyle=\color{gitlightblue},
    numberstyle=\color{gray},
    xleftmargin=10mm,
    numbers=left,
    numbersep=10mm,
    breakatwhitespace=false,         
    breaklines=true,                 
    captionpos=b,                    
    keepspaces=true,                 
    showspaces=false,                
    showstringspaces=false,
    showtabs=false,                  
    tabsize=2
}

\lstdefinestyle{C++}{
    language=C++,
    basicstyle=\color{black}\ttfamily\small,
    commentstyle=\color{gray},
    keywordstyle=\color{gitred},
    stringstyle=\color{gitlightblue},
    numberstyle=\color{gray},
    xleftmargin=10mm,
    numbers=left,
    numbersep=10mm,
    breakatwhitespace=false,         
    breaklines=true,                 
    captionpos=b,                    
    keepspaces=true,                 
    showspaces=false,                
    showstringspaces=false,
    showtabs=false,                  
    tabsize=2
}

\lstdefinestyle{C}{
    language=C,
    basicstyle=\color{black}\ttfamily\small,
    commentstyle=\color{gray},
    keywordstyle=\color{gitred},
    stringstyle=\color{gitlightblue},
    numberstyle=\color{gray},
    xleftmargin=10mm,
    numbers=left,
    numbersep=10mm,
    breakatwhitespace=false,         
    breaklines=true,                 
    captionpos=b,                    
    keepspaces=true,                 
    showspaces=false,                
    showstringspaces=false,
    showtabs=false,                  
    tabsize=2
}


% Code blocks
\usepackage{newfloat}
\DeclareFloatingEnvironment[
    fileext=loc,
    listname={List of Codes},
    name=Code,   % The name that will appear in the caption
    % placement=H, % H: here, t: top, b: bottom
    placement=h,
    within=none,
]{codeblock}


% Code boxes
\usepackage{tcolorbox}
\tcbuselibrary{listings, skins}
\newtcblisting{code}[3][]{
    enhanced,
    listing only,
    listing options={style=#2},
    colback=gitgray,
    colframe=gitgray, % frame color
    boxrule=0pt,      % frame thickness
    arc=0mm,          % corner radius,
    #1
    title=\small#3,         % title
    coltitle=black,   % title color
    fonttitle=\bfseries, % title font
    attach boxed title to top left={yshift=-0.7mm,xshift=0mm},
    boxed title style={colback=gitgray,boxrule=0pt,bottomrule=1pt},
    underlay unbroken={\fill[gitgray] (title.north east) rectangle (frame.north east);},
    underlay unbroken and first={\draw[black,line width=0.5pt] (title.north west)-- ($(title.north west-|frame.east)$);},
    underlay unbroken and first={\draw[black,line width=0.5pt] (frame.north west)-- (frame.north east);},
    underlay unbroken and last={\draw[black,line width=0.5pt] (frame.south west)-- (frame.south east);},
}

% Code-only
% Command \begin{codeonly}{language}...\end{codeonly}
% to display only the code without a box, 
% just the listings code syntax highlighting
% depending on the language in input
\lstnewenvironment{codeonly}[1]{\lstset{style=#1}}{}

% Inline code
% Define a new command for inline code
\newcommand{\cc}[1]{
    \texttt{#1}
    % \kern-0.7ex
    % \tcbox[colback=gitgray,
    %        coltext=black,
    %        on line, 
    %        boxrule=0pt, 
    %        boxsep=0.5pt,
    %        top=0.2mm, 
    %        bottom=0.2mm, 
    %        left=0.5mm, 
    %        right=0.5mm]
    % {\texttt{#1}}
    % \kern-0.7ex
}