\documentclass{article}

% --------------------------------- PREAMBLE -----------------------------------

% Importing preamble settings from settings/ folder
\usepackage{import}
\subimport{settings/}{general.tex}
\subimport{settings/}{colors.tex}
\subimport{settings/}{figures.tex}
\subimport{settings/}{math.tex}
\subimport{settings/}{toc.tex}
\subimport{settings/}{code.tex}
\subimport{settings/}{fonts.tex}
% \subimport{settings/}{callouts.tex} emoji, requires lualatex

% Additional preamble options
\graphicspath{{images/}}

% Additional preable commands
\usepackage{changepage}

% Subfiles package (best loaded last in the preamble)
\usepackage{subfiles}

% --------------------------------- SETTINGS -----------------------------------

% Title, Author and Date
\title{Modeling of MPI Collective Operations:\\ Broadcast and Reduce}
\author{Marco Tallone}
\date{March 2024}

% --------------------------------- DOCUMENT -----------------------------------

% Document
\begin{document}

\maketitle

\begin{abstract}
\noindent
Collective operations are the building blocks of parallel applications. For this reason, there is an interest for efficient implementations of these operations. This work focuses on the broadcast and reduce MPI implementations and presents two model for latency prediction of these operations: one based on point-to-point communications and the other on a linear regression model.
The models are compared among each other and with the actual performance of the MPI implementations both in terms of explainability and overall accuracy to understand the trade-offs between the two approaches.
\end{abstract}

\section{Introduction}\label{introduction}
\subfile{sections/introduction}

\section{Collective and Point-to-Point Communications}\label{algorithms}
\subfile{sections/algorithms}

\vspace{-5mm}
\section{Epyc Architecture and Latency Measurements}\label{architecture}
\subfile{sections/architecture}

\section{Point-to-Point Communications Model}\label{model}
\subfile{sections/model}

\section{Linear Model}\label{linear}
\subfile{sections/fit}

\pagebreak
\section{Results}\label{results}
\subfile{sections/results}

\pagebreak
\section{Conclusions}\label{conclusions}
\subfile{sections/conclusions}

% Bibliography
\pagebreak
\bibliographystyle{plainurl}
\bibliography{bibliography}

% Appendix
\pagebreak
\appendix
\renewcommand{\thesection}{\Alph{section}} % Change section numbering to A, B, C, ...
\renewcommand{\thesubsection}{\thesection\arabic{subsection}} % Change subsection numbering to A1, A2, ...
\section{Appendix}\label{appendix}
\subfile{sections/appendix}

\end{document}